%%%%%%%%%%%%%%%%%%%%%%%%%%%%%%%%%%%%%%%%%
%
% Author: Catarina Silva
% Email: c.alexandracorreia@ua.pt; c.alexandracorreia@av.it.pt
% Version: 1
%
%----------------------------------------------------------------------------------------

\documentclass[11pt]{scrartcl} % Font size

\input{structure.tex}

\title{	
	\normalfont\normalsize
	\textsc{Universidade de Aveiro}\\
	\vspace{25pt} % Whitespace
	\rule{\linewidth}{0.5pt}\\ % Thin top horizontal rule
	\vspace{20pt} % Whitespace
	{\huge Wireless Sensor Network: overhead and cipher/decipher time}\\ % The assignment title
	\vspace{12pt} % Whitespace
	\rule{\linewidth}{2pt}\\ % Thick bottom horizontal rule
	\vspace{12pt} % Whitespace
}

\author{\LARGE Catarina Silva \\ \large Redes Móveis, 2020-2021} % Your name

\date{\normalsize\today} % Today's date (\today) or a custom date

\begin{document}

\maketitle % Print the title

\section{Introduction}

Wireless Sensor Networks (WSN) have been studied to enable different applications such as environmental surveillance, scientific observation, traffic monitoring and it can
be developed to Smart Homes~\cite{liu2009passive, 7029932}. WSN is comprised of a large number of sensor nodes that are deployed in an unattended and remote environment 
to measure a few physiological parameters. Typically, it consists of a large number of resource-limited sensor nodes working in a self-organizing and distributed manner.

Limitations in processing power, battery life, communication bandwidth and memory constrain the applicability of existing cryptography standards for small embedded 
devices~\cite{ganesan2003analyzing}.

The algorithms are in table~\ref{table:t00} have been suggested as a good algorithm for sensor networks. 

\begin{table}[h]
	\centering
	\caption{Encryption Schemes and Parameters}
	\label{table:t00}
	\begin{tabular}{llll}
		\toprule
		\textbf{Algorithm} & \textbf{Type} & \textbf{Key/Hash} & \textbf{Block} \\
		\midrule
		RC4 & stream & 128 bits & 8 bits \\
		IDEA & block & 128 bits & 64 bits \\
		RC5 & block & 64 bits & 64 bits \\
		MD5 & 1-way hash & 128 bits & 512 bits \\
		SHA1 & 1-way hash & 128 bits & 512 bits \\
		\bottomrule
	\end{tabular}
\end{table}

\textbf{RC4} is a stream cipher symmetric key algorithm. This algorithm is quite simple and operations involve the addition of 8 bit elements or swapping variables in a 256-byte 
state table. RC4 supports variable length keys.

\textbf{IDEA} (International Data Encryption Algorithm) is a symmetric-key block cipher that operates on 64 bit plaintext blocks. The key is 128 bits long with the same algorithm 
used for both encryption and decryption. The algorithm primarily includes operations from three algebraic groups: XOR, addition modulo 216, multiplication modulo 216+1.

\textbf{RC5} is a fast symmetric block cipher with a variety of parameters: block size, key size and number of rounds. We currently focus on a RC5 implementation with a 64-bit data 
block and 64-bit key. It uses the XOR, addition and rotation operations.

\textbf{MD5} is a one-way hash function that processes the input text in 512 bit blocks to generate a 128-bit hash value. The mathematical operations that are involved in this 
algorithm are: XOR, AND, OR, NOT and rotations. The algorithm also pads plaintext to 512 blocks with the last 64 bits of the last block indicating the length of the
message.

\textbf{SHA-1} is also a one-way hash function that produces a 160-bit output when any message of any length less than 264 bits is input. The operations are similar to MD5 and 
constitute XOR, AND, OR, NOT and rotations. 

%Apenas para guardar: https://www.paessler.com/network-analyzer-diagnostics

\section{Task 1}

\section{Task 2}

\section{Conclusion}

\bibliographystyle{unsrt}
\bibliography{refs}{}

\end{document}
